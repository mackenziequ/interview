\documentclass[answers]{exam}
\usepackage{amsmath, fullpage}
\usepackage{amssymb}
\usepackage{graphicx}

\begin{document}


\title{Interview Questions and Solutions}
\author{Mackenzie Qu}
%\date{\vspace{-5ex}}
\maketitle
\thispagestyle{empty}
\section{Regression Analysis}
\begin{questions}
%Regression Theory
\question Suppose you wanted to model $Y$ using $X$, and you decided to use the linear regression:
\[
  Y = X \beta + \varepsilon
  \text{.}
\]
What assumptions are being made?
How would you find $\beta$?
What tests can be done on $\beta$ afterwards?

\begin{solution}[.2in]
    \\
    The genearl assumptions for linear regression are:
    \begin{itemize}
        \item Weak exogeneity: The variables are treated as fixed values with no 
        measurement errors. Meaning that the error term is uncorrelated with the independent variables.
        \item Linearity: The relationship between the independent and the dependent
     variables is linear, shown by the mean of the dependent variable is a linear combination
     of the parameters and dependent variables. However, the assumptions of linearity is only restricting the parameters.
     Transformation(like polynomial regression) techniques can be used to fit datasets.
        \item No perfect multicollinearity(for multiple regression): The independent variables are not correlate to each other. The common problems includes overfitting.
        \item Constant Variance(homoscedasticity): The variance of the errors does not depend on the value of
        the predictor variables.
        \item Independence of errors: The errors of the dependent variables are uncorrelated of each other.
        \item Error term distribution: The error terms are generally assumed to be Normally distributed $N(\mu, \sigma^2)$, but other distributions can be used. 
    \end{itemize}
    To find $\beta$, we can use sum of squared method or likelihood method

\end{solution}

\question %Regression Theory
What are the significance tests used for the parameters estimated in a logistic regression?
How are they different than those used for a linear regression?

\question
%Regression Theory
What are the assumptions required for a linear regression?
What is multicollinearity, and what are its implications?
How would you measure goodness of fit?

\question
%Regression Theory
What assumptions are needed for a linear regression?
Are they the same for a logistic regression?
How would you test the significance of the parameters for a linear regression?
Would you use the same test in the case of a logistic regression?
\end{questions}

\section{Probability and Statistics}
\begin{questions}
    
\question
500 ants are randomly put on a 1 metre string. 
Each ant randomly moves toward on end of the string with equal 
probability to the left or the right at constant speed of 1 m/min 
until it falls off one end of the string. 
Also assume that the size of the ant is infinitely small. 
When two ants collide head-on, they both immediately change directions 
and keep on moving at 1 m/min. 
What is the expected time for all ants to fall off the string?

\begin{solution}
    Consider the case where there is only 1 ant on the string. The ant will 
have a position of $d_l$, denoting the distance to the left and $d_r$ denoting 
the distance to the right, such that $d_l+d_r = 1$.\newline 
Let $t_1$ be the time for the ant to fall off the rope. To find
 the expected time that the aunt will take to fall off, 
 we are
looking for $E(t_1)$, where
$$E(t_1) = E(\frac{\text{distance the ant will travel to fall off}}{\text{the volocity the ant is travelling at}})=\frac{E(\text{distance the ant will travel to fall off})}{1}$$
as the the volocity of ant is constant. 
\begin{align*}
    E(\text{distance the ant will travel to fall off})=&
    P(\text{travelling to the left direction})\times
    \text{distance to the left}+\\
    &P(\text{travelling to the right direction})\times
    \text{distance to the right}\\
    =&\frac{1}{2}\times d_l + \frac{1}{2} \times d_r\\
    =&\frac{1}{2}\times (d_l+d_r)
    =\frac{1}{2}
\end{align*}
\\
Now consider the case where there are two ants. When two ants collide into each other,
the momentum of the two ants carry over - we can imagine it as the two ants have swapped positions and
continued going in the same direction.\\
In this case, we are looking for the expected value of the maximum amout of time each ant spends on the rope.
ie. $E(max\{t_1, t_2\})$, which is equivalent to finding the maximum distance an ant is from the edge of the rope.\\
Since each ant is randomly dropped onto the rope, the distance $D_i$ of each ant toward the direction it is moving is 
independent and identically distributed on $Uniform(0,1)$. 
Let $Y = max\{D_1, D_2\}$, the cumulative distribution function of $Y$ follows:
\begin{align*}
    F_Y(y) &= P(Y<y)\\
    &= P(max\{D_1, D_2\}<y)\\
    &= P(D_1<y)P(D_2<y) \dots \text{because of independence}\\
    &= F_{D_1}(y)*F_{D_2}(y)\\
    &= y^2
\end{align*}
and the probability density function follows:
$$f_Y(y) = \frac{d}{dy}F_Y(y) = 2y$$
$$E(Y) = \int_{0}^{1} y * 2y\ dy = \int_{0}^{1}2y^2\ dy = \frac{2}{3}$$
\\
To find the answer for 500 ants, let $Y = max\{D_1,..., D_{500}\}$. It follows:
\begin{align*}
    F_Y(y) &= y^{500},\\
    f_y(y) &= 500*y^{499},\\
    E(Y) &= \frac{500}{501}
\end{align*}

\end{solution}

\question
There are 50 noodles in a bowl. 
You can tie two ends of either one noodle or two different noodles, forming a nod.
After 50 nods are tied. What is the expected value of number of circles in the bowl?
\question
There are two dice. The game is as following: 
Two dice will be thrown at once. 
Adding the number together you will get a number ranging from 2 to 12.
The probability of getting 12 is 40\%, and 6\% for getting 2,3,4,5,...11
Person A and B will need to guess what the number would be. 
The one who has a closer guess will win.
A will guess first and then B guess, after they've guessed, 
two dice will be thrown at once, 
what number should A guess and what number should be B guess?
\end{questions}

\end{document}